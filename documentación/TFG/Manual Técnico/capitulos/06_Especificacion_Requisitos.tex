\chapter{Especificación de requisitos}\label{cap:especificacion_requisitos}

\section{Introducción}

Se desea desarrollar una aplicación web que permita la la gestión de comisiones de los centros pertenecientes a la Universidad de Córdoba. Las siguientes secciones describen los actores del sistema (Sección \ref{sec:actores}), la descripción modular (Sección \ref{sec:modulos}), los requisitos del sistema (Sección  \ref{sec:requisitos-del-sistema})  y los supuestos semánticos de la información que se van a utilizar (Sección \ref{sec:supuestos-semanticos}).

\section{Actores del sistema}\label{sec:actores}

Se van a considerar los siguientes tipos de usuario en la aplicación web: 
    \begin{itemize}
    \item Público: usuario no registrado en el sistema que podrá consultar la información pública.
    \item Usuario universitario: usuario registrado en el sistema que podrá obtener distintos certificados históricos y actuales sobre la participación como miembro de equipo de gobierno, junta o comisiones a la largo del tiempo.
    \item Responsable de la comisión: usuario registrado en el sistema que será responsable de la gestión de la información de la comisión responsable.
    \item Responsable de junta: usuario registrado en el sistema que será responsable de la gestión de la información de la junta responsable y de todas sus comisiones.
    \item Responsable de centro: usuario registrado en el sistema que será responsable de la gestión de la información todas las comisiones de su centro.
    \item Administrador: usuario registrado en el sistema que tendrá un control completo de la aplicación. En particular, tendrá las competencias exclusivas de la gestión de todos los tipos de usuarios, centros y comisiones. Además, se encargará de la gestión de las copias de seguridad.    
    \end{itemize}

\section{Módulos de la aplicación}\label{sec:modulos}

La aplicación web va a estar compuesta por cuatro módulos que corresponderán a cada uno de los tipos de usuario considerados y que se describen en las siguientes secciones.

\subsection{Módulo del usuario público}

El usuario público podrá consultar toda la información pública que esté disponible:
\begin{itemize}
    \item Consultar la información pública: comisiones de un centro, miembros actuales de una comisión, consulta de actas aprobadas y pendientes de aprobación,...
    \item Consultar la ayuda del usuario público.
\end{itemize}

\subsection{Módulo del usuario universitario}

Además de las acciones disponibles para el usuario público, el usuario universitario podrá desarrollar las siguientes actividades:
\begin{itemize}
    \item Obtención de diferentes tipos certificados (pertenencia a una comisión, comisiones en las cuales ha pertenecido en un rango de fechas...).
    \item Consultar la ayuda del miembro de la comisión.
\end{itemize}

\subsection{Módulo del responsable de la comisión}
Además de las acciones disponibles para el usuario universitario, el responsable de la comisión podrá desarrollar las siguientes actividades:

\begin{itemize}
    \item Consultar, insertar, actualizar y eliminar la información de la comisión responsable.
    \item Asignar/excluir miembros de las comisiones responsables.
    \item Consultar, insertar, actualizar y eliminar la información de las convocatorias de la comisión responsable.
    \item Consultar la ayuda del responsable de comisión.
\end{itemize}

\subsection{Módulo del responsable de junta}
Además de las acciones disponibles para el responsable de la comisión, el responsable de la junta podrá desarrollar las siguientes actividades:
\begin{itemize}
    \item Consultar, insertar, actualizar y eliminar la información de la junta responsable.
    \item Consultar, insertar, actualizar y eliminar la información de las comisiones de la junta.
    \item Asignar/excluir miembros de las juntas del centro responsable.
    \item Asignar/excluir responsable de las comisiones de la junta responsable.
    \item Consultar, insertar, actualizar y eliminar la información de las convocatorias de la junta responsable.
    \item Consultar la ayuda del responsable de junta.
\end{itemize}

\subsection{Módulo del responsable de centro}
Además de las acciones disponibles para el responsable de la junta, el responsable del centro podrá desarrollar las siguientes actividades:
\begin{itemize}
    \item Consultar, insertar, actualizar y eliminar la información del centro responsable.
    \item Consultar, insertar, actualizar y eliminar la información de las juntas del centro.
    \item Asignar/excluir miembros de gobierno del centro responsable.
    \item Asignar/excluir responsable de las juntas del centro responsable.
\item Consultar la ayuda del responsable de centro.
\end{itemize}

\subsection{Módulo del administrador}

El usuario de tipo Administrador tendrá un control total de la aplicación. Además, se encargará de forma exclusiva de los siguientes módulos:
\begin{itemize}
    \item Gestión de usuarios registrados: responsables y miembros de centro, junta y comisión.
    \item Gestión de los centros universitarios.
    \item Gestión de copias de seguridad.
    \item Consultar la ayuda del administrador.
\end{itemize}

En principio, la aplicación solamente tendrá un único administrador.

\section{Requisitos del sistema}\label{sec:requisitos-del-sistema}

Los requisitos del sistema hacen referencia a todas las características relacionadas con la aplicación web. Se describirán los siguientes tipos de requisitos:
\begin{itemize}
    \item Requisitos funcionales: describen las tareas que la aplicación debe realizar para satisfacer las necesidades del problema (Sección \ref{sec:requisitos-funcionales}).
    \item Requisitos no funcionales: describen cómo se tiene que satisfacer las necesidades del problema  (Sección \ref{sec:requisitos-no-funcionales}).
    \item Requisitos de la interfaz: describen cómo debe ser la comunicación entre el usuario y la aplicación  (Sección \ref{sec:requisitos-interfaz}).
    \item Requisitos de la información: describen las características de los datos que se van a gestionar (Sección \ref{sec:requisitos-información}).
\end{itemize}


\subsection{Requisitos funcionales}\label{sec:requisitos-funcionales}

Los requisitos funcionales indican lo que el sistema debe hacer. Cada uno de estos requisitos debe tener dos propiedades: 
\begin{itemize}
    \item Ser completo: el requisito debe mencionar exactamente lo que el sistema debe hacer
    \item Ser cerrado: el requisito debe ser claro y no estar abierto a múltiples interpretaciones, sino solamente a una.
\end{itemize}

Los requisitos funcionales que se van a considerar se agruparán según los módulos de los tipos de usuario y se denotarán como RF-<nº requisito>.

 \begin{itemize}
 \item \textbf{Módulo del usuario público}
 \item[] La aplicación debe permitir que el usuario público pueda realizar las siguientes acciones:
     \begin{itemize}
         \item RF-1. Consultar la información vigente de los miembros del equipo de gobierno de todos los centros de la UCO.
         \item RF-2. Consultar la información vigente de los miembros de las juntas de todos los centros de la UCO.
         \item RF-3. Consultar la información vigente de las comisiones pertenecientes a las juntas de todos los centros de la UCO.
         \item RF-4. Consultar la información vigente de los miembros de las comisiones de las comisiones pertenecientes a las juntas de todos los centros de la UCO.
         \item RF-5. Consultar la información de las convocatorias realizadas de las juntas vigentes de todos los centros de la UCO.
         \item RF-6. Consultar la información de las convocatorias realizadas de las comisiones vigentes de las juntas de todos los centros de la UCO.
         \item RF-7. Consultar la ayuda para el usuario público. 
     \end{itemize}

 \item \textbf{Módulo del usuario universitario}
 \item[] La aplicación debe permitir que el usuario universitario registrado pueda realizar las siguientes acciones: 
     \begin{itemize}
         \item RF-8. Gestionar la información de su perfil.
             \begin{itemize}
                 \item RF-8.1. Modificar la constraseña del usuario.
                 \item RF-8.2. Modificar imagen de usuario.
             \end{itemize}               
         \item RF-9. Obtener diferentes tipos de certificados:
             \begin{itemize}
                 \item RF-9.1. Descarga de certificado de situación actual.
                 \item RF-9.2. Descarga de certificado de centros en los que ha participado como miembro de equipo de gobierno en un periodo de tiempo.
                 \item RF-9.3. Descarga de certificado de certificado de juntas que ha representado en un periodo de tiempo.
                 \item RF-9.4. Descarga de certificado de comisiones a las que ha pertenecido en un periodo de tiempo.
                  \item RF-9.4. Descarga de certificado de convocatorias de junta o comisión a las que ha asistido en un periodo de tiempo.
             \end{itemize}            
         \item RF-10. Consultar la ayuda para usuario universitario.
     \end{itemize}

 \item \textbf{Módulo del responsable de comisión}
 \item[] La aplicación debe permitir que el responsable de comisión en activo pueda realizar las siguientes acciones:
     \begin{itemize}
         \item RF-11. Gestionar los miembros de la comisión responsable.
             \begin{itemize}
                  \item RF-11.1. Crear un miembro de comisión.
                  \item RF-11.2. Buscar un miembro de comisión.
                  \item RF-11.3. Consultar un miembro de comisión.
                  \item RF-11.4. Modificar un miembro de comisión.
                  \item RF-11.5. Eliminar un miembro de comisión.
                  \item RF-11.6. Asignar/desasignar miembro responsable de comisión.
             \end{itemize} 
         \item RF-12. Gestionar las convocatorias de la comisión.
             \begin{itemize}
                  \item RF-12.1. Crear una convocatoria de comisión.
                  \item RF-12.2. Buscar una convocatoria de comisión.
                  \item RF-12.3. Consultar una convocatoria de comisión.
                  \item RF-12.4. Modificar una convocatoria de comisión.
                  \item RF-12.5. Eliminar una convocatoria de comisión.
                  \item RF-12.6. Asignar miembros a una convocatoria de comisión.
                  \item RF-12.7. Desasignar miembros una convocatoria de comisión.
             \end{itemize} 
         \item RF-13. Consultar la ayuda para el responsable de comisión.
     \end{itemize}

\item \textbf{Módulo del responsable de junta}
 \item[] La aplicación debe permitir que el responsable de junta en activo pueda realizar las siguientes acciones:
     \begin{itemize}
        \item RF-14. Gestionar los miembros de la junta responsable.
             \begin{itemize}
                  \item RF-14.1. Crear un miembro de junta.
                  \item RF-14.2. Buscar un miembro de junta.
                  \item RF-14.3. Consultar un miembro de junta.
                  \item RF-14.4. Modificar un miembro de junta.
                  \item RF-14.5. Eliminar un miembro de junta.
                  \item RF-14.6. Asignar/desasignar miembro responsable de junta.
             \end{itemize}  
         \item RF-15. Gestionar las  comisiones de la junta responsable.
             \begin{itemize}
                  \item RF-15.1. Crear una comisión.
                  \item RF-15.2. Buscar una comisión.
                  \item RF-15.3. Consultar una comisión.
                  \item RF-15.4. Modificar una comisión.
                  \item RF-15.5. Eliminar una comisión.
             \end{itemize}    
          \item RF-16. Gestionar los miembros de todas las comisiones de la junta responsable.
             \begin{itemize}
                  \item RF-16.1. Crear un miembro de comisión.
                  \item RF-16.2. Buscar un miembro de comisión.
                  \item RF-16.3. Consultar un miembro de comisión.
                  \item RF-16.4. Modificar un miembro de comisión.
                  \item RF-16.5. Eliminar un miembro de comisión.
                  \item RF-16.6. Asignar/desasignar miembro responsable de comisión.
             \end{itemize} 
             
             \item RF-17. Gestionar las convocatorias de la junta responsable.
             \begin{itemize}
                  \item RF-17.1. Crear una convocatoria de junta.
                  \item RF-17.2. Buscar una convocatoria de junta.
                  \item RF-17.3. Consultar una convocatoria de junta.
                  \item RF-17.4. Modificar una convocatoria de junta.
                  \item RF-17.5. Eliminar una convocatoria de junta.
                  \item RF-17.6. Asignar miembros a una convocatoria de junta.
                  \item RF-17.7. Desasignar miembros una convocatoria de junta.
             \end{itemize}
          \item RF-18. Gestionar las convocatorias de todas las comisiones de la junta responsable.
             \begin{itemize}
                  \item RF-18.1. Crear una convocatoria de comisión.
                  \item RF-18.2. Buscar una convocatoria de comisión.
                  \item RF-18.3. Consultar una convocatoria de comisión.
                  \item RF-18.4. Modificar una convocatoria de comisión.
                  \item RF-18.5. Eliminar una convocatoria de comisión.
                  \item RF-18.6. Asignar miembros a una convocatoria de comisión.
                  \item RF-18.7. Desasignar miembros una convocatoria de comisión.
             \end{itemize} 
         \item RF-19. Consultar la ayuda para el responsable de junta.
     \end{itemize}

\item \textbf{Módulo del responsable de centro}
 \item[] La aplicación debe permitir que el responsable de centro en activo pueda realizar las siguientes acciones:
     \begin{itemize}
         \item RF-20. Gestionar los miembros de gobierno del centro responsable.
             \begin{itemize}
                  \item RF-20.1. Crear un miembro de gobierno.
                  \item RF-20.2. Buscar un miembro de gobierno.
                  \item RF-20.3. Consultar un miembro de gobierno.
                  \item RF-20.4. Modificar un miembro de gobierno.
                  \item RF-20.5. Eliminar un miembro de gobierno.
                 \item RF-20.6. Asignar/desasignar miembro responsable de centro.
             \end{itemize} 
        \item RF-21. Gestionar las juntas del centro responsable.
             \begin{itemize}
                  \item RF-21.1. Crear una junta.
                  \item RF-21.2. Buscar una junta.
                  \item RF-21.3. Consultar una junta.
                  \item RF-21.4. Modificar una junta.
                  \item RF-21.5. Eliminar una junta.
             \end{itemize} 
        \item RF-22. Gestionar los miembros todas las juntas del centro responsable.
             \begin{itemize}
                  \item RF-22.1. Crear un miembro de junta.
                  \item RF-22.2. Buscar un miembro de junta.
                  \item RF-22.3. Consultar un miembro de junta.
                  \item RF-22.4. Modificar un miembro de junta.
                  \item RF-22.5. Eliminar un miembro de junta.
                  \item RF-22.6. Asignar/desasignar miembro responsable de junta.
             \end{itemize}  
         \item RF-23. Gestionar las convocatorias de todas las juntas del centro responsable.
             \begin{itemize}
                  \item RF-23.1. Crear una convocatoria de junta.
                  \item RF-23.2. Buscar una convocatoria de junta.
                  \item RF-23.3. Consultar una convocatoria de junta.
                  \item RF-23.4. Modificar una convocatoria de junta.
                  \item RF-23.5. Eliminar una convocatoria de junta.
                  \item RF-23.6. Asignar miembros a una convocatoria de junta.
                  \item RF-23.7. Desasignar miembros una convocatoria de junta.
             \end{itemize}
        \item RF-24. Gestionar todas las comisiones de todas las juntas del centro responsable.
             \begin{itemize}
                  \item RF-24.1. Crear una comisión.
                  \item RF-24.2. Buscar una comisión.
                  \item RF-24.3. Consultar una comisión.
                  \item RF-24.4. Modificar una comisión.
                  \item RF-24.5. Eliminar una comisión.
             \end{itemize}    
          \item RF-25. Gestionar los miembros de todas las comisiones de todas las juntas del centro responsable.
             \begin{itemize}
                  \item RF-25.1. Crear un miembro de comisión.
                  \item RF-25.2. Buscar un miembro de comisión.
                  \item RF-25.3. Consultar un miembro de comisión.
                  \item RF-25.4. Modificar un miembro de comisión.
                  \item RF-25.5. Eliminar un miembro de comisión.
                  \item RF-25.6. Asignar/desasignar miembro responsable de comisión.
             \end{itemize} 
        \item RF-26. Gestionar las convocatorias de todas las comisiones de todas las juntas del centro responsable.
             \begin{itemize}
                  \item RF-26.1. Crear una convocatoria de comisión.
                  \item RF-26.2. Buscar una convocatoria de comisión.
                  \item RF-26.3. Consultar una convocatoria de comisión.
                  \item RF-26.4. Modificar una convocatoria de comisión.
                  \item RF-26.5. Eliminar una convocatoria de comisión.
                  \item RF-26.6. Asignar miembros a una convocatoria de comisión.
                  \item RF-26.7. Desasignar miembros una convocatoria de comisión.
             \end{itemize}
         \item RF-27. Consultar la ayuda para el responsable de centro.
     \end{itemize}

 \item \textbf{Módulo del administrador}
 \item[] La aplicación debe permitir que el administrador pueda realizar las siguientes acciones:
     \begin{itemize}
        \item RF-28. Gestionar los centros de la UCO.
             \begin{itemize}
                  \item RF-28.1. Crear un centro.
                  \item RF-28.2. Buscar un centro.
                  \item RF-28.3. Consultar un centro.
                  \item RF-28.4. Modificar un centro.
                  \item RF-28.5. Eliminar un centro.
             \end{itemize} 
        \item RF-29. Gestionar los miembros de gobierno de todos los centros.
             \begin{itemize}
                  \item RF-29.1. Crear un miembro de gobierno.
                  \item RF-29.2. Buscar un miembro de gobierno.
                  \item RF-29.3. Consultar un miembro de gobierno.
                  \item RF-29.4. Modificar un miembro de gobierno.
                  \item RF-29.5. Eliminar un miembro de gobierno.
                 \item RF-29.6. Asignar/desasignar miembro responsable de centro.
             \end{itemize} 
         \item RF-30. Gestionar las juntas de todos los centros.
             \begin{itemize}
                  \item RF-30.1. Crear una junta.
                  \item RF-30.2. Buscar una junta.
                  \item RF-30.3. Consultar una junta.
                  \item RF-30.4. Modificar una junta.
                  \item RF-30.5. Eliminar una junta.
             \end{itemize} 
        \item RF-31. Gestionar los miembros todas las juntas de todos los centros.
             \begin{itemize}
                  \item RF-31.1. Crear un miembro de junta.
                  \item RF-31.2. Buscar un miembro de junta.
                  \item RF-31.3. Consultar un miembro de junta.
                  \item RF-31.4. Modificar un miembro de junta.
                  \item RF-31.5. Eliminar un miembro de junta.
                  \item RF-31.6. Asignar/desasignar miembro responsable de junta.
             \end{itemize}
        \item RF-32. Gestionar las convocatorias de todas las juntas de todos los centros.
             \begin{itemize}
                  \item RF-32.1. Crear una convocatoria de junta.
                  \item RF-32.2. Buscar una convocatoria de junta.
                  \item RF-32.3. Consultar una convocatoria de junta.
                  \item RF-32.4. Modificar una convocatoria de junta.
                  \item RF-32.5. Eliminar una convocatoria de junta.
                  \item RF-32.6. Asignar miembros a una convocatoria de junta.
                  \item RF-32.7. Desasignar miembros una convocatoria de junta.
             \end{itemize}
        \item RF-33. Gestionar todas las comisiones de todas las juntas de todos los centros.
             \begin{itemize}
                  \item RF-33.1. Crear una comisión.
                  \item RF-33.2. Buscar una comisión.
                  \item RF-33.3. Consultar una comisión.
                  \item RF-33.4. Modificar una comisión.
                  \item RF-33.5. Eliminar una comisión.
             \end{itemize}    
          \item RF-34. Gestionar los miembros de todas las comisiones de todas las juntas de todos los centros.
             \begin{itemize}
                  \item RF-34.1. Crear un miembro de comisión.
                  \item RF-34.2. Buscar un miembro de comisión.
                  \item RF-34.3. Consultar un miembro de comisión.
                  \item RF-34.4. Modificar un miembro de comisión.
                  \item RF-34.5. Eliminar un miembro de comisión.
                  \item RF-34.6. Asignar/desasignar miembro responsable de comisión.
             \end{itemize} 
        \item RF-35. Gestionar las convocatorias de todas las comisiones de todas las juntas de todos los centros.
             \begin{itemize}
                  \item RF-35.1. Crear una convocatoria de comisión.
                  \item RF-35.2. Buscar una convocatoria de comisión.
                  \item RF-35.3. Consultar una convocatoria de comisión.
                  \item RF-35.4. Modificar una convocatoria de comisión.
                  \item RF-35.5. Eliminar una convocatoria de comisión.
                  \item RF-35.6. Asignar miembros a una convocatoria de comisión.
                  \item RF-35.7. Desasignar miembros una convocatoria de comisión.
             \end{itemize}
         \item RF-36. Consultar la ayuda para el administrador.
         \item RF-37. Gestionar los usuario de la UCO.
             \begin{itemize}
                  \item RF-37.1. Crear un usuario.
                  \item RF-37.2. Buscar un usuario.
                  \item RF-37.3. Consultar un usuario.
                  \item RF-37.4. Modificar un usuario.
                  \item RF-37.5. Eliminar un usuario.
             \end{itemize} 
     \end{itemize}
 \end{itemize}

\subsection{Requisitos no funcionales}\label{sec:requisitos-no-funcionales}

Los requisitos no funcionales representan cómo tiene que trabajar la aplicación. Los requisitos no funcionales se denotarán como RNF-<nº requisito>.

\begin{itemize}
    \item RNF-1. La aplicación debe tener una interfaz gráfica que sea fácil de usar y amigable para el usuario.
    \item RNF-2. La aplicación debe ser robusta y adaptable a cualquier dispositivo.
    \item RNF-3. La aplicación deberá funcionar correctamente en los principales navegadores web.
    \item RNF-4. Los usuarios registrados en el sistema deberán identificarse con nombre de usuario y contraseña para acceder.
    \item RNF-5. La aplicación solamente tendrá un usuario con el rol de administrador.
    \item RNF-6. El borrado de todos registros en la base de datos serán borrados lógicos (soft delete), mediante actualización de un campo.
\end{itemize}


\subsection{Requisitos de la interfaz}\label{sec:requisitos-interfaz}
  
  La interfaz es el dispositivo que permite la comunicación entre el usuario y el sistema. En esta sección, se enumeran los requisitos que debe tener la interfaz para que pueda ser utilizada por todos los tipos de usuario.
  
  Los requisitos la interfaz  especifican cómo deber la comunicación entre el usuario y la parte visible de la aplicación. Se denotarán como RINT-<nº de requisito> 


\begin{itemize}
    \item RINT-1. La interfaz debe ser gráfica, intuitiva, sencilla y agradable para el usuario.
    \item RINT-2. La nomenclatura que usará la interfaz para mostrar la información y las distintas opciones al usuario será la más clara posible.
    \item RINT-3.La interfaz utilizará menús para mostrar la información correspondiente a cada tipo de usuario.    
    \item RINT-4.La interfaz utilizará menús para mostrar la información correspondiente a cada tipo de usuario.
    \item RINT-5. La interfaz utilizará diferentes tipos de mensajes: informativos, de error, de confirmación, etc.
    \item RINT-6. La interfaz utilizará formularios para pedir al usuario la información que corresponda, resaltando aquellos campos que sean obligatorios.
\end{itemize}

\subsection{Requisitos de información}\label{sec:requisitos-información}

Los requisitos de información hacen referencia a los datos que debe gestionar la aplicación web. Se denotarán como RI-<no de requisito>.

Se deberá almacenar la siguiente información:
\begin{itemize}
    \item RI-1. Usuarios registrados: responsables de centros, de juntas, de comisiones y usuarios universitarios. Para cada usuario, se deberá almacenar como mínimo su nombre, email y contraseña.
    \item RI-2. Centros: facultades, escuelas y centros adscritos.
    \item RI-3. Juntas: fecha de constitución, fecha disolución.
    \item RI-4. Comisiones: comisión de Asuntos Económicos, comisión de Docencia, comisión de Ordenación Académica, comisión de Relaciones Exteriores, comisión de Reconocimientos y Convalidaciones, comisión Académica de los Másteres, unidades de Garantía de Calidad...
    \item RI-5. Convocatorias: ordinaria, extraordinaria, urgente.
    \item RI-6. Miembros de gobierno: director, secretario, vicedirector, subdirector.
    \item RI-7. Miembros de junta: Profesorado vinculación permanente, otro personal docente e investigador y  PAS, alumnado, personal libre designación.
    \item RI-8. Miembros de comisión: Profesorado vinculación permanente, otro personal docente e investigador y  PAS, alumnado, personal libre designación.
\end{itemize}

        
Una descripción más detallada de la información que se va a gestionar se puede consultar en la sección \ref{sec:supuestos-semanticos} de Supuestos Semánticos y en el capítulo \ref{cap:modelo_de_datos} de Modelo de Datos.

 \section{Supuestos semánticos}\label{sec:supuestos-semanticos}

Una vez descritos los requisitos funcionales, de información y de interfaz, los supuestos semánticos que se definen a continuación describen de manera más formal y precisa las restricciones identificadas en la definición del problema.
