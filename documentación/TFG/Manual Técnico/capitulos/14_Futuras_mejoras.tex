\chapter{Futuras mejoras}

\section{Introducción}
El sistema desarrollado en este trabajo es susceptible de mejoras y ampliaciones, como cualquier otro software. De hecho, se identificaron muchas oportunidades durante el desarrollo.

Las mejoras contempladas se pueden agrupar en diferentes vertientes según su propósito o naturaleza:

\begin{itemize}
    \item Mejoras y nuevas funciones. Esta categoría incluye cambios que hacen que el software sea más eficiente, fácil de usar o funcional.
    
    \item Integración con otras aplicaciones. El desarrollo de nuevas aplicaciones en la Sección de Soporte u otros entornos similares puede dar lugar a la implementación de conectividades entre todos los sistemas. Esto contribuiría a una mayor eficiencia en el desarrollo del trabajo.
    
    \item Capacidad de extensión y exportación. Además, es importante considerar la posibilidad de exportar este sistema a otras unidades organizativas de la universidad o de otras organizaciones.
\end{itemize}

En todo caso, el hecho de que se aborden las mejoras aquí propuestas, en su totalidad o en parte, dependerá, en última instancia, de la continuidad de su uso, puesto que, tal como se indicó en la sección 3.4, el sistema desarrollado tendrá un uso provisional.

En las siguientes secciones se desglosan las mejoras y ampliaciones propuestas, según cada una de las categoría expuestas.

\section{Mejoras y nuevas funciones}
Basándonos en lo que se ha desarrollado hasta ahora, se identifican una serie de funciones adicionales y mejoras que se pueden realizar. Estas mejoras se describen a continuación, divididas por tipo de usuario y mejoras generales.

\subsection{Mejoras para usuarios}
\begin{itemize}
    \item Generar más tipos de certificados, con más información detallada
    \item COn respecto al usuario público, poder consultar información en un periodo de tiempo de las comisiones de la Universidad de Córdoba, en lugar de solamente las actuales.
\end{itemize}

\subsection{Mejoras para responsables}
\begin{itemize}
    \item Añadir más opciones de trámites automatizados (correos a terceros, por ejemplo).
    \item Permitir el tratamiento masivo de miembros. Por ejemplo, crear a la vez los miembros de junta de una junta en concreto, siendo la fecha de toma de posesión la misma, ya que normalmente al crearse una junta la toma de posesión de la mayoría de los miembros es el mismo día.
\end{itemize}

\subsection{Mejoras para administradores}
\begin{itemize}
    \item Añadir más opciones de filtros por cada tipo de entidad para una mayor precisión de localización del elemento a buscar.
    \item Generar estadísticas sobre el uso de la aplicación.
    \item Añadir funcionalidad de supervisión de las acciones realizadas por diferentes roles, como por ejemplo, consultar porcentaje de absentismo de asistencia a las convocatorias de comisión y junta.
\end{itemize}

\subsection{Mejoras de funcionamiento}
\begin{itemize}
    \item Hacer más eficiente el proceso de gestión de las paradas de tramitación y el cálculo de los tiempos de tramitación.
    \item Establecer y registrar los motivos por los que un trabajo se cierra o termina, para facilitar la comprensión y el seguimiento del trabajo realizado.
    \item Permitir que los usuarios puedan personalizar la aplicación, ya que su uso podría extenderse en el tiempo y algunos apartados podrían cambiar.
\end{itemize}

\subsection{Integración con otras aplicaciones}
Se propone integrar la aplicación con otras aplicaciones que se utilizan actualmente o que puedan utilizarse en el futuro por el personal de la Universidad de Córdoba.

Se pudo observar en una visita al Rectorado de Córdoba, como se utilizaban múltiples aplicaciones web separadas por funcionalidad, teniendo cada una de ellas un acceso diferente, duplicado y poco conectado entre ellas.

Una gran mejora sería la de unificar todas ellas en un sistema en conjunto como si de la Intranet de la Universidad de Córdoba se tratara. Actualmente se pudo observar como el personal laboral tenía en el navegador web en una lista de favoritos enlaces a cada una de las aplicaciones web desarrolladas para la gestión de cada tarea.

\subsection{Exportabilidad}
Para que la aplicación pueda exportarse a otros entornos, se deben definir como variables de entorno los elementos que dependen de la ubicación o configuración específica del entorno deseado.

\begin{itemize}
    \item Conectores y cuentas sociadas
    \item Ubicación y denominación de orígenes de datos
\end{itemize}
