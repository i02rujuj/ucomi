\chapter{Conclusiones}

\section{Introducción}
En este capítulo se exponen la conclusiones obtenidas sobre el desarrollo del proyecto, y en relación con los objetivos planteados inicialmente en el capítulo 2 frente al resultado final alcanzado tras la finalización del mismo y las pruebas realizadas. 

Con carácter general, el objetivo principal, consistente en el desarrollo de un sistema informático integrado para la gestión de comisiones de la Universidad de Córdoba, se ha cumplido satisfactoriamente, logrando una solución que anteriormente no existía.

\section{Conclusiones específicas}

\section{Conclusiones personales y profesionales}
A nivel personal, el presente trabajo me ha permitido reciclar, actualizar y adquirir nuevos conocimientos técnicos, especialmente en el uso de Frameworks como Laravel. Estos Frameworks facilitan el desarrollo de software escalable y eficiente, lo que permitirá abordar nuevas funcionalidades de manera segura en el futuro.

En cuanto al ámbito profesional, se han logrado los objetivos propuestos:

\begin{itemize}

\item La Universidad de Córdoba dispone ahora de una herramienta personalizada para el desarrollo de su trabajo, que ha demostrado, a lo largo de todo el periodo de pruebas, cubrir suficientemente las necesidades planteadas, sin perjuicio de las mejoras que se proponen en el capítulo 14.

\item La experiencia acumulada durante el desarrollo de este trabajo permite asentar conocimientos sobre el desarrollo de aplicaciones web, así como la utilización de las tecnologías que están a la orden del día, permitiendo con ello un constante aprendizaje continuo en esta materia. 

\end{itemize}


