\chapter{Objetivos}
\section{Objetivo principal}
 El objetivo de este Trabajo de Fin de Grado es el desarrollo de una aplicación web para la gestión de las comisiones de los centros de la Universidad de Córdoba: facultades y escuelas superiores. En particular, se desea gestionar la información de sus juntas de centro o juntas de escuela y de las comisiones delegadas de las mismas, como las comisiones de docencia, planes de estudios, asuntos económicos, etc.

\section{Objetivos específicos}

Los objetivos específicos de este Trabajo de Fin de Grado son los siguientes:

\begin{itemize}
    \item \textbf{Tipos de usuarios}
    \item[] Se deberán permitir los siguientes tipos de usuarios: 
     \begin{itemize}
        \item Administrador
          \begin{itemize}
              \item Este tipo de usuario estará registrado en el sistema y tendrá un control completo de la aplicación. 
              \item En particular, tendrá las competencias exclusivas de la gestión de todos los tipos de usuarios, centros, juntas, miembros y comisiones. Además, se encargará de la gestión de las copias de seguridad.
          \end{itemize}
          \item Responsable del centro
           \begin{itemize}
               \item Este tipo de usuario estará registrado en el sistema y representará a una persona que podrá gestionar la información de un centro.
               \item En particular, tendrá las competencias exclusivas de la asignación/exclusión a los miembros de gobierno pertenecientes a dicho centro, de creación de las juntas que pertenezcan al centro, así como asignación/exclusión del responsable de cada junta.
            \end{itemize}
            \item Responsable de la junta
           \begin{itemize}
               \item Este tipo de usuario estará registrado en el sistema y representará a una persona que podrá gestionar la información de una junta de centro.
               \item En particular, tendrá las competencias exclusivas de la asignación/exclusión a los miembros de junta pertenecientes a dicha junta, de la gestión de las convocatorias realizadas por la junta, de los miembros que participarán en cada una de las convocatorias de la junta, de creación de las comisiones que pertenezcan a la junta, así como asignación/exclusión del responsable de cada comisión.
            \end{itemize}
          \item Responsable de la comisión 
           \begin{itemize}
               \item Este tipo de usuario estará registrado en el sistema y representará a una persona que podrá gestionar la información de una comisión.
               \item En particular, tendrá las competencias exclusivas de la asignación/exclusión a los miembros pertenecientes a dicha comisión, gestión de las convocatorias realizadas por la comisión, y de los miembros que participarán en cada una de las convocatorias de dicha comisión.
            \end{itemize}
        \item Usuario universitario
            \begin{itemize}
                \item Este tipo de usuario estará registrado en el sistema.
                \item Podrá obtener distintos certificados como pueden ser de situación actual, de centros en los que ha participado como miembro de equipo de gobierno, de juntas a las que ha representado, de comisiones a las que ha pertenecido, de convocatorias en las que ha participado...
              \end{itemize}
        \item Público
        \begin{itemize}
              \item Este tipo de usuario no necesitará estar registrado en el sistema.
              \item Podrá consultar la información pública disponible: comisiones de un centro, miembros actuales de una comisión, consulta de actas aprobadas y pendientes de aprobación,...
          \end{itemize}
     \end{itemize}
    \item \textbf{Bases de datos relacional}
     \item[] Se deberá diseñar una base de datos relacional que permita gestionar toda la información relacionada con los centros, juntas, comisiones, convocatorias y miembros de gobierno, junta y comisión:
      \begin{itemize}
            \item Usuarios registrados: administrador, responsable centro, responsable junta comisión y usuario registrado.
            \item Centros docentes de la Universidad de Córdoba: Facultades y Escuelas.
            \item Juntas de centro o facultad: fecha constitución, fecha disolución.
            \item Comisiones: 
                \begin{itemize}
                    \item Comisión de Asuntos Económicos.
                    \item Comisión de Docencia.
                    \item Comisión de Ordenación Académica.
                    \item Comisión de Relaciones Exteriores.
                    \item Comisión de Reconocimientos y Convalidaciones.
                    \item Comisión Académica de los Másteres.
                    \item Unidades de Garantía de Calidad.
                    \item Etc.
                \end{itemize}
          \item Convocatorias de juntas y comisiones: sesión ordinaria, sesión extraordinaria.
          \item Miembro de gobierno, junta y comisión: Director, Vicedirector, profesorado permanente, PAS, alumnado...
      \end{itemize}
    \item \textbf{Módulos}
     \item[] Se deberán diseñar los siguiente módulos principales: 
        \begin{itemize}
            \item Módulo del administrador
                \begin{itemize}
                    \item Gestión de usuarios registrados: responsables y miembros de la comisión
                    \item Gestión de los centros universitarios.
                    \item Gestión de copias de seguridad
                    \item Consultar la ayuda del administrador.
                \end{itemize}

            \item Módulo del responsable de centro
                \begin{itemize}
                    \item Consultar, insertar, actualizar y eliminar la información del centro responsable.
                    \item Consultar, insertar, actualizar y eliminar la información de las juntas del centro.
                    \item Asignar/excluir miembros de gobierno del centro responsable.
                    \item Asignar/excluir responsable de las juntas del centro responsable.
                \item Consultar la ayuda del responsable de centro.
                \end{itemize}

             \item Módulo del responsable de junta
                \begin{itemize}
                    \item Consultar, insertar, actualizar y eliminar la información de la junta responsable.
                    \item Consultar, insertar, actualizar y eliminar la información de las comisiones de la junta.
                    \item Asignar/excluir miembros de las juntas del centro responsable.
                    \item Asignar/excluir responsable de las comisiones de la junta responsable.
                    \item Consultar, insertar, actualizar y eliminar la información de las convocatorias de la junta responsable.
                \item Consultar la ayuda del responsable de junta.
                \end{itemize}

             \item Módulo del responsable de comisión
                \begin{itemize}
                    \item Consultar, insertar, actualizar y eliminar la información de la comisión responsable.
                    \item Asignar/excluir miembros de las comisiones responsables.
                    \item Consultar, insertar, actualizar y eliminar la información de las convocatorias de la comisión responsable.
                \item Consultar la ayuda del responsable de comisión.
                \end{itemize}

            \item Módulo del usuario universitario.
                \begin{itemize}
                    \item Obtención de diferentes tipos certificados (pertenencia a una comisión, comisiones en las cuales ha pertenecido en un rango de fechas...).
                \item Consultar la ayuda del usuario universitario.
                \end{itemize}
         
         \item Módulo público
            \begin{itemize}
                \item Consultar la información pública: equipo de gobierno de un centro, juntas de un centro, comisiones de una junta, miembros actuales de gobierno, junta, de una comisión,...
                \item Consultar la ayuda del usuario público.
            \end{itemize}
        
        \end{itemize}
     
    \item \textbf{Diseño de la interfaz}
    \item[] Se deberá diseñar una interfaz intuitiva, robusta, amigable y adaptable a los distintos navegadores web.

    \item \textbf{Documentación}
    \item[] Se deberá redactar la siguiente documentación: manual técnico, manual de usuario y manual de código.
\end{itemize}

    
\section{Objetivos personales}

Se proponen los siguientes objetivos personales:
 \begin{itemize}
    \item Aprender y afianzar conocimientos sobre el entorno de desarrollo web Wamp \cite{wamp}, que permite instalar y configurar un servidor web en un entorno local que será utilizado como entorno de desarrollo y testeo de nuestra aplicación web.
    \item Aprender y afianzar conocimientos sobre la contratación y configuración de del \textit{hosting} VPS \cite{vps}, que será utilizado como entorno de producción, así como la utilización y configuración de dicho \textit{hosting} haciendo uso del panel de control Plesk \cite{plesk}.
    \item Aprender y afianzar conocimientos sobre el sistema de gestión de versiones Git \cite{git} para controlar todos nuestros cambios de la aplicación web en un repositorio local y remoto.
    \item Aprender y afianzar conocimientos sobre las herramientas y recursos que nos brinda el \textit{framework} denominado ``Laravel'' \cite{laravel} para el desarrollo de aplicaciones web modernas.
    \item  Aprender y afianzar conocimientos sobre el funcionamiento del gestor de bases de datos MySql \cite{mysql}.
    \item Familiarizarse con la herramienta Overleaf \cite{overleaf} para la edición de documentos en \LaTeX.   
     \item  Aprender y afianzar conocimientos sobre el \textit{framework} ''Tailwind CSS'' \cite{tailwind} para el diseño de la aplicación web.
     \item  Aprender y afianzar conocimientos sobre diferentes librerías como ''Orgchart JS'' \cite{orgchart} para la creación de organigramas interactivos.
 \end{itemize}
    
    
    