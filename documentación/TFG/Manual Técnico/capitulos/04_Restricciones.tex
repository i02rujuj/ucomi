\chapter{Restricciones} \label{cap:restricciones}

\section{Introducción}

Este capítulo va a describir las restricciones o limitaciones que van a afectar al desarrollo de la aplicación web. 

Las restricciones se pueden clasificar en dos categorías:
\begin{itemize}
    \item Factores iniciales: hacen referencia a la limitaciones impuestas por la propia definición del problema. Estas restricciones no pueden modificarse o eliminarse.
    \item Factores estratégicos: hacen referencia a las diferentes alternativas que se van a considerar para el desarrollo de la aplicación web. En cada caso, se deberá indicar la opción elegida y los motivos de su elección.
\end{itemize}


\section{Factores iniciales} \label{sec:iniciales}

Los factores iniciales,  también conocidos `como ``factores dato'', son limitaciones que que vienen dadas por la naturaleza del problema real. Este tipo de restricciones no se pueden  modificar bajo ningún concepto, ya que entonces eso implicaría abordar un problema nuevo.

Se van a considerar los siguientes factores iniciales:
\begin{itemize}
    \item Se debe desarrollar una aplicación web.
    \item La aplicación web se debe ejecutar en cualquier navegador.
    \item La interfaz debe cumplir con un mínimo de principios de usabilidad que permita ser efectiva, útil, intuitiva, con una estructura consistente, sencilla en su navegación y adaptable a cualquier dispositivo.
    \item Debe contemplar los módulos del usuario público, responsable de la comisión, miembro de la comisión y administrador. 
    \item Se debe garantizar que los datos solamente sean accedidos por aquellos usuarios que tengan la la autorización correspondiente.
    \item El código deberá ser modular para facilitar el mantenimiento y actualización de la aplicación web.
\end{itemize}

Una descripción más detallada de los módulos de los usuarios se mostrará en el Capítulo \ref{cap:especificacion_requisitos} de Especificación de Requisitos.


\section{Factores estratégicos} \label{sec:estrategicos}

Para el desarrollo de la aplicación web, se van a considerar los siguientes factores estratégicos, indicando las alternativas disponibles en cada caso y la opción elegida y los motivos de su elección.


\subsection{Metodología}


Se han considerado diferentes metodologías para el desarrollo de la aplicación web:
\begin{itemize}
    \item  Paradigma de desarrollo de software clásico o ``Modelo en Cascada'' \cite{pressman}:  se caracteriza por el desarrollo lineal de cada una de las fases del proceso de desarrollo. Actualmente, esta metodología se encuentra obsoleta porque no permite desarrollar distintas fases del proyecto sin pasar por todas las anteriores.
    \item Metodología OMT (\textit{Object-Modeling Technique}) \cite{omt}, también denominada ``Paradigma de modelado y diseño orientado a objetos'': esta metodología no se centra en las tareas de cada módulo, sino en la definición de los objetos que se utilicen describiendo sus métodos y atributos. También está en desuso.
    \item Metodologías ágiles como \textit{Scrum} \cite{scrum}, que utiliza iteraciones que desarrollan partes del sistema en periodos cortos de tiempo.
    \item Metodología basada en la notación UML (\textit{Unified modified language})\cite{uml} o ``Lenguaje Unificado de Modelado'':  es un lenguaje de modelado bien definido con una notación estándar para el desarrollo de aplicaciones de software que está amparado por la compañía \textit{Object Management Group} (OMG) \cite{omg}. 
\end{itemize}

Para el proceso de abstracción del mundo real a nuestra aplicación, se utilizarán los diagramas del lenguaje de modelado unificado de sus siglas del inglés UML, la metodología basada en la notación de UML que permite un modelado visual que se comprende con facilidad. Además, está aprobada como estándar internacional por la \textit{International Organization for Standardization} bajo la norma ISO/IEC 19501:2005 \cite{iso19501}. UML proporciona numerosos diagramas para la descripción del proceso de desarrollo del software, como, por ejemplo, el diagrama de casos de uso, que refleja los actores del sistema y los escenarios se ejecutarán, el diagrama de secuencia que muestra cómo el programa realiza una funcionalidad, o el diagrama de clases que refleja las distintas clases, sus propiedades y métodos.


\subsection{Entorno de desarrollo}
El entorno de desarrollo integrado (IDE\footnote{En inglés, \textit{Integrated Development Environment} (IDE).}) es la herramienta utilizada para desarrollar la aplicación. Se han considerado varias alternativas, como son Visual Studio Code \cite{visualStudioCode}, editor de código con múltiples y potentes extensiones, Eclipse \cite{eclipse} o Sublime Text \cite{sublimeText}.

Se ha elegido Visual Studio Code \cite{visualStudioCode} por tener soporte para múltiples lenguajes de programación y contar con extensiones que ayudaran a trabajar aún más rápido.

Como \textit{framework} se ha elegido Laravel\cite{laravel} porque tiene herramientas que facilitan el desarrollo de aplicaciones web e incluye funcionalidades extra como las relacionadas con la seguridad.

\subsection{Base de datos}\label{sec:bases-de-datos}

Laravel\cite{laravel} permite el uso de diferentes sistemas de gestión de bases de datos, como MySql \cite{mysql}, Oracle \cite{oracle} o PostgreSQL \cite{postgreSQL}. Se ha elegido MySql \cite{mysql} por los siguientes motivos:
\begin{itemize}
    \item Software de código abierto y multiplataforma.
    \item Posee una gran conectividad, velocidad y seguridad, siendo muy apropiado para acceder a bases de datos en Internet.
    \item Es uno de los sistemas de gestión de base datos más populares del mundo, sobre todo para el desarrollo de aplicaciones web.
\end{itemize}


\subsection{Lenguaje de programación}

Existen numerosos lenguajes de programación que se pueden utilizar para el desarrollo de aplicaciones web, como PHP \cite{PHP},  Python \cite{python} o Ruby \cite{ruby}. Se ha elegido el lenguaje de programación PHP \cite{PHP} porque es el que utiliza Laravel \cite{laravel}, el \textit{framework} seleccionado. PHP es un lenguaje multiplataforma, interpretado y orientados a objetos, que puede combinarse con código HTML \cite{html}, lenguaje para hojas de estilo CSS \cite{css} y código JavaScript \cite{javascript} para el desarrollo de aplicaciones web.

    