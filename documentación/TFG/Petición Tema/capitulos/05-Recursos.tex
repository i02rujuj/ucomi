\chapter{Recursos}\label{cap:recursos}

\section{Recursos humanos}
\begin{itemize}
    \item \textbf{Autor}
    \item[] Javier Ruiz Jurado, estudiante del Grado en Ingeniería Informática, especialidad de Ingeniería del Software.
    \item \textbf{Director}
    \item[] José Luis Ávila Jiménez
\end{itemize}

\section{Recursos de hardware}

Para el desarrollo de la aplicación en el entorno local, se va a utilizar el equipo del alumno que tiene las características siguientes:
\begin{itemize}
    \item Ordenador: Acer Nitro 5
    \item Sistema operativo: Windows 11 Professional
    \item RAM instalada: 8GB DDR4
    \item Procesador: Intel(R) Core(TM) i7-10750H CPU @ 2.60GHz 
\end{itemize}

Para el despliegue de la aplicación web, se hará uso de un servidor VPS con las características siguientes:
\begin{itemize}
    \item Platform: Linux x86-64
    \item OS Package: Centos 7 (for AMD64/Intel EM64T) Virtuozzo Template
    \item Memory: 4.00 GB
    \item SSD: 50.00 GB
\end{itemize}


\section{Recursos de software}

Se van a utilizar los siguientes recursos de software para el desarrollo de la aplicación web: 
\begin{itemize}
    \item Editor de código fuente para el desarrollo de la aplicación web: Visual Studio Code \cite{visualStudioCode}
    \item \textit{Framework} utilizado para el desarrollo de la aplicación web, tanto en \textit{Back end} y \textit{Front end}: Laravel \cite{laravel}.
    \item Sistema de gestión de base de datos: MySql \cite{mysql}.
    \item Lenguajes de Programación: PHP \cite{PHP}, JavaScript \cite{javascript}.
    \item Redacción del documento: Overleaf \cite{overleaf}, editor en línea de documentos escritos en \LaTeX. 
    \item Editor de diagramas: draw.io \cite{drawio} y StarUML \cite{starUML}.
    \item Repositorio remoto en Github \cite{github}.
\end{itemize}


